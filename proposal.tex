\documentclass{article}
\usepackage[utf8]{inputenc}
\usepackage{graphicx}
\usepackage{color}
\usepackage{listings}

\usepackage{color}

\usepackage{amsmath}
\usepackage{graphicx}
\usepackage{amssymb}

\definecolor{dkgreen}{rgb}{0,0.6,0}
\definecolor{gray}{rgb}{0.5,0.5,0.5}
\definecolor{mauve}{rgb}{0.58,0,0.82}


\lstset{frame=tb,
  language=Java,
  aboveskip=3mm,
  belowskip=3mm,
  showstringspaces=false,
  columns=flexible,
  basicstyle={\small\ttfamily},
  numbers=none,
  numberstyle=\tiny\color{gray},
  keywordstyle=\color{blue},
  commentstyle=\color{dkgreen},
  stringstyle=\color{mauve},
  breaklines=true,
  breakatwhitespace=true,
  tabsize=3
}
\renewcommand{\refname}{Bibliografia}




\title{CSC 210 Project Proposal}
\author{Cengiz Ozel, Christina Liu, Evan Cohen-Doty}
\date{November, 1, 2021}

\begin{document}

\maketitle

\section{Description}
Our application will be an online banking thing. Someone add fancier things here


\section{Satisfying the Requirements}

Here we'll just outline, probably in detail, the different aspects of the application which satisfy his list of minimum requirements.

\section{What Makes Our Project Special}

Some bullshit here, title is self-explanatory.

\section{Who Did What}

Here we'll just say who did what, probably best if each person writes their own subsection detailing what they did so they can talk it up.
\subsection{Cengiz}

\subsection{Christina}

\subsection{Evan}

\section{Link to Github}

We can just place a link here, maybe if it's a bit big, we'll explain the layout for ease of access.


\end{document}
